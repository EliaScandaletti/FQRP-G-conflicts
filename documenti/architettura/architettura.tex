\documentclass[../main.tex]{subfiles}

\begin{document}

L'architettura adottata prevede l'utilizzo di una pipeline costituita da varie componenti definite nel namespace \texttt{core}.
All'interno del namespace \texttt{fqrp} queste componenti vengono implementate in modo diverso.

All'interno del \texttt{main} è possibile, quindi, assemblare la pipeline scegliendo l'implementazione di ciascun componente.

\subsection{Namespace core}
\subfile{sections/core.tex}

\subsection{Namespace fqrp}
\subfile{sections/fqrp.tex}
% \subparagraph{Filtri}
% Particolare rilevanza ha il generatore \texttt{FilteredGenerator} che permette di applicare un filtro a un altro generatore.
% Nello specifico, questo generatore contiene un altro generatore a cui richiede dati che ritorna solo 

\end{document}