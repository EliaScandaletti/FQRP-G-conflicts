\documentclass[dvipsnames]{article}

\usepackage{amsfonts}
\usepackage{amsmath}
\usepackage{amsthm}
\usepackage{cases}
\usepackage{float}
\usepackage{graphicx}
\usepackage{listings}
\usepackage{pgfplots}
\usepackage{pxfonts}
\usepackage{subcaption}
\usepackage{subfiles}
\usepackage{xcolor}

\usetikzlibrary{plotmarks}

\theoremstyle{definition}
\newtheorem{definition}{Definizione}

\definecolor{backcolour}{rgb}{0.95,0.95,0.92}

\lstdefinelanguage{ps}{
    morekeywords=[1]{function, true},
    morekeywords=[2]{set, while, for, return, if, else},
    morekeywords=[3]{assert, swap, max, min, random, randomInt, randomFrom, push, pop},
    morecomment=[l]{//},
    sensitive=true
}
\lstset{
    backgroundcolor=\color{backcolour},
    keywordstyle=[1]\color{Blue},
    keywordstyle=[2]\color{DarkOrchid},
    keywordstyle=[3]\color{Blue},
    numberstyle=\scriptsize\color{darkgray},
    commentstyle=\color{ForestGreen},
    basicstyle=\ttfamily\small,
    numbers=left,
    numbersep=5pt,
    language=ps
}

\pgfplotsset{compat=1.18}
\pgfplotsset{colormap name=viridis}

\newcommand{\makeaxis}[2]{
    \begin{minipage}{0.4\paperwidth}
        \begin{tikzpicture}
            \begin{axis} [
                title={#1}
            ]
                \addplot+ [
                    scatter,
                    mesh,
                    empty line=jump,
                    mark=square*,
                    restrict expr to domain={{\thisrow{freq}}{1:{+inf}}},
                    scatter src={\thisrow{freq}/\thisrow{max_freq}}
                    ] table [x=n,y=value] {#2};
            \end{axis}
        \end{tikzpicture}
    \end{minipage}
}

\newcommand{\makefigure}[2]{
    \begin{figure}[H]
        \makebox[\textwidth]{
            \expandafter\makeaxis{Generatore esaustivo}{data/exh_gen_#2.dat}
            \hfill
            \expandafter\makeaxis{Generatore esaustivo con flitro}{data/exh_filt_gen_#2.dat}
        }
        \vskip\baselineskip
        \makebox[\textwidth]{
            \expandafter\makeaxis{Generatore casuale con filtro}{data/rand_filt_gen_#2.dat}
            \hfill
            \expandafter\makeaxis{Generatore a intervalli}{data/int_filt_gen_#2.dat}
        }
        \vskip\baselineskip
        \makebox[\textwidth]{
            \pgfplotscolorbardrawstandalone[
                colorbar horizontal,
                point meta min=0,
                point meta max=100,
                colorbar style={
                    width=\textwidth,
                    xtick={0,10,...,100},
                    xticklabel={\pgfmathprintnumber{\tick}\%}
                }
            ]
        }
        \caption{#1}
    \end{figure}
}

\renewcommand{\makefigure}[2]{Figures have been disabled for compilation time reason. Comment this line to activate plots.}

\title{Relazione di Tirocinio Interno}
\author{Elia Scandaletti}
\date{Maggio - Giugno 2022}

\begin{document}

\maketitle
\tableofcontents

\section*{TODO list}
Elenco di cose da fare per completare la relazione:
\begin{itemize}
    \item scrivere una conclusione migliore alla sezione sulla prob di conflitto di tipo C e vedere se si riesce a tirar fuori qualcosa di utile;
    \item riportare nella parte teorica le osservazioni fatte per la scrittura del codice:
    \begin{itemize}
        \item distanza tra partenza e destinazione di veicoli in catena di conflitti di tipo C strettamente crescente;
        \item altro? ricontrollare il codice per trovare altre osservazioni interessanti.
    \end{itemize}
    \item fare un confronto tra generatori casuale con filtro e generatore a intervalli e tra generatori casuali e generatore esaustivi;
    \item finire sezione su architettura scelta;
    \item scrivere qualcosa nella sezione dei risultati numerici;
    \item trovare una rappresentazione migliori per i grafici.
\end{itemize}

\section{Analisi dei Conflitti nei FQRP-G}
\subfile{conflitti/conflitti.tex}

\section{Metodi per la generazione casuale di istanze FQRP-G non partizionate}
\subfile{generazione_istanze/generazione_istanze.tex}

\section{Architettura scelta}
\subfile{architettura/architettura.tex}

\section{Risultati ottenuti}
\subfile{risultati/risultati.tex}

\end{document}