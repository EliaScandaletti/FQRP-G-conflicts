\documentclass[../../../relazione.tex]{subfiles}

\begin{document}

Questo metodo è il più intuitivo e facile da implementare.
Al contempo è anche il meno efficiente.

Il metodo consiste nel generare istanze casuali (per esempio, con l'algoritmo di Fisher-Yates) e, successivamente, considerare solo le istanze non partizionate.

Questo metodo è poco efficiente e richiede che vengano generate intere istanze che poi verranno scartate.

\paragraph{Implementazione d'esempio}
\lstinputlisting{alg/casuale_filtro_post.ps}

\paragraph{Efficienza nella generazione}
% TODO

\paragraph{Distribuzione delle permutazioni}
% TODO

\end{document}