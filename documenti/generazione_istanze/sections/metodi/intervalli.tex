\documentclass[../../../relazione.tex]{subfiles}

\begin{document}

Questo metodo prevede di:
\begin{enumerate}
    \item scegliere una destinazione casuale per il veicolo 1 t.c. $1 < \pi_1 \leq n$;
    \item considerare $[1,\, \pi_1]$ come massimo intervallo non partizionato;
    \item scegliere casualmente se il prossimo veicolo andrà verso destra o verso sinistra;
          \label{alg:gen_int:desc:scelta}
    \item se il veicolo andrà verso destra:
          \begin{enumerate}
              \item scegliere una destinazione $\pi_k$ al di fuori del massimo intervallo non partizionato
                    \footnote{si noti che poiché il percorso di ogni veicolo appartiene al massimo intervallo non partizionato, allora scegliere una destinazione all'infuori di tale intervallo garantisce che tale destinazione sia libera};
              \item scegliere una partenza $k$ non ancora scelta nel massimo intervallo non partizionato;
              \item considerare $[1,\, \pi_k]$ come massimo intervallo non partizionato;
          \end{enumerate}
    \item se il veicolo andrà verso sinistra:
          \begin{enumerate}
              \item scegliere una partenza $k$ al di fuori del massimo intervallo non partizionato
                    \footnote{si noti che poiché il percorso di ogni veicolo appartiene al massimo intervallo non partizionato, allora scegliere una partenza all'infuori di tale intervallo garantisce che tale partenza sia libera};
              \item scegliere una destinazione $\pi_k$ non ancora scelta nel massimo intervallo non partizionato;
              \item considerare $[1,\, k]$ come massimo intervallo non partizionato;
          \end{enumerate}
    \item se il massimo intervallo non partizionato coincide con $[1,\, n]$, procedere al prossimo punto;
          altrimenti tornare al punto \ref{alg:gen_int:desc:scelta});
    \item associare in modo casuale le rimanenti partenze con le rimanenti destinazioni.
\end{enumerate}

Si noti che al punto \ref{alg:gen_int:desc:scelta}) è possibile adottare varie strategie per la scelta della direzione del veicolo.
Quella più intuitiva è considerare ciascuna direzione equiprobabile.
Questa scelta deriva dal fatto che a ogni iterazione all'esterno del massimo intervallo non partizionato ci sono tante partenze libere quante destinazioni libere.
Analogamente, questa proprietà vale anche all'interno dell'intervallo.

Il rischio che questo metodo introduce è la possibile presenza di bias.
Maggiori approfondimenti sono richiesti.

\paragraph{Implementazione d'esempio}
\lstinputlisting{alg/intervalli.ps}

\paragraph{Efficienza nella generazione}
% TODO

\paragraph{Distribuzione delle permutazioni}
% TODO

\end{document}