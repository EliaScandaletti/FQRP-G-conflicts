\documentclass[../../relazione.tex]{subfiles}

\begin{document}

\subsection{Introduzione}
In questa sezione vengono esposti i risultati numerici ottenuti.

Per ogni metrica misurata, per ogni generatore utilizzato, vengono proposti due grafici.

Il primo grafico rappresenta la distribuzione della metrica in funzione della dimensione del problema.
Ogni punto indica che quel valore è possibile.
Il colore del punto indica qualitativamente la probabilità che quel valore venga trovato.
In particolare, il colore indica il rapporto tra frequenza del valore rappresentato e la frequenza del valore più frequente in istanze della stessa dimensione.
In questo modo, ogni colonna del grafico avrà uno o più valori con frequenza massima e rapporto uguale a 1, e gli altri valori avranno un rapporto compreso tra 0 e 1.

Inoltre, la linea rossa rappresenta il valore medio della metrica in funzione della dimensione dell'istanza e le barre verticali rappresentano il primo e il terzo quartile.

Nel secondo grafico, viene riportata la funzione di distribuzione della metrica.
Ogni colore rappresenta una dimensione dell'istanza.

\subsection{Generatori utilizzati}
I dati sono stati ricavati utilizzando quattro generatori diversi:
\begin{itemize}
    \item Generatore esaustivo: genera tutte le possibili istanze;
    \item Generatore esaustivo con filtro: genera tutte le possibili istanze escludendo quelle partizionate e considerando solo una volta quelle simmetriche;
    \item Generatore casuale con filtro: applica l'algoritmo discusso nella sezione \ref{alg_gen:rand_filt};
    \item Generatore a intervalli: applica l'algoritmo discusso nella sezione \ref{alg_gen:int}.
\end{itemize}

\subsection{Conflitti di arco}
\makefigure{Conflitti di arco}{arcType}

\subsection{Conflitti di tipo A}
\makefigure{Conflitti di tipo A}{AType}

\subsection{Conflitti di tipo B}
\makefigure{Conflitti di tipo B}{BType}

\subsection{Conflitti di tipo C}
\subsubsection{Numero di conflitti}
\makefigure{Numero di conflitti}{CGraph_arcs_num}

\subsubsection{Numero di catene}
\makefigure{Numero di catene}{CGraph_chain_num}

\subsubsection{Lunghezza della catena massima}
\makefigure{Lunghezza della catena massima}{CGraph_max_length}

\subsubsection{Numero di alberi di conflitto}
\makefigure{Numero di alberi di conflitto}{CGraph_tree_num}

\subsubsection{Numero di veicoli coinvolti}
\makefigure{Numero di veicoli coinvolti}{CGraph_vehicles_num}

\subsection{Conflitti di tipo misto}
\subsubsection{Numero di conflitti}
\makefigure{Numero di conflitti}{MForest_edges_num}

\subsubsection{Dimensione della componente connessa di conflitti più grande}
\makefigure{Dimensione della componente connessa di conflitti più grande}{MForest_max_tree_size}

\subsubsection{Numero di veicoli coinvolti}
\makefigure{Numero di veicoli coinvolti}{MForest_nodes_num}

\subsubsection{Numero di alberi}
\makefigure{Numero di alberi}{MForest_tree_num}

\end{document}