\documentclass[../../../../relazione.tex]{subfiles}

\begin{document}

Questo metodo è simile al precedente.
Consiste nel verificare se un'istanza è partizionata \textit{mentre} viene generate casualmente.

In questo modo, non appena una partizione viene rilevata, l'algoritmo non perde tempo per completare la generazione.
Così facendo, l'algoritmo diventa più efficiente rispetto al precedente.

\paragraph{Implementazione d'esempio}
\lstinputlisting{alg/casuale_filtro_parz.ps}

\paragraph{Efficienza nella generazione}
% TODO

\paragraph{Distribuzione delle permutazioni}
% TODO

\end{document}