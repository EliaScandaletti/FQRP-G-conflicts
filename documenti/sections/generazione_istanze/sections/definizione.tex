\documentclass[../../../relazione.tex]{subfiles}

\begin{document}

\begin{definition}\label{def_part}
  Un'istanza FQRP è detta partizionata se è possibile partizionare i veicoli di tale istanza in parti tali per cui i cammini minimi di due veicoli appartenenti a parti distinte non possono intersecarsi.
\end{definition}
Formalmente, i cammini minimi di due veicoli $i$ e $j$ appartenenti a parti distinte possono intersecarsi sse
\[ \min(j,\, \pi_j) \leq \max(i,\, \pi_i)\, \wedge\, \min(i,\, \pi_i) \leq \max(j,\, \pi_j) \]
dove $\pi$ è la permutazione degli arrivi rispetto alle partenze e $\pi_k$ è la destinazione del veicolo k.

In altri termini, un'istanza è partizionata se è possibile partizionare la corrispondente permutazione $\pi$ in modo tale che
\begin{equation}
  \label{cond_part}
  \max(i,\, \pi_i) < \min(j,\, \pi_j),\, \forall\, i,\, j : i < j \mbox{ appartenenti a parti distinte}.
\end{equation}

In questa definizione, e nel resto del documento, dove non diversamente specificato, si considerano i veicoli e gli indici di $\pi$ da 1 a $n$.

\end{document}