\documentclass[../../../relazione.tex]{subfiles}

\begin{document}

Basandosi sulla definizione (\ref{def:nodeC}), è possibile definire una catena di conflitti.
Parliamo di una catena di conflitti quando un veicolo è oggetto di un conflitto di tipo C e è a sua volta soggetto di un altro conflitto dello stesso tipo.
Per esempio, dati i veicoli $i$, $j$ e $k$ e i conflitti $i(j)$ e $j(k)$, allora $i$, $j$ e $k$ formano una catena di conflitto.

\end{document}