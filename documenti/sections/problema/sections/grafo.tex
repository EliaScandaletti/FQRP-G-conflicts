\documentclass[../../../relazione.tex]{subfiles}

\begin{document}

È possibile rappresentare i conflitti di tipo C di un'istanza del problema in un grafo orientato i cui vertici rappresentano i veicoli e i archi rappresentano un nodo.
In particolare, il vertice $(i, j)$ rappresenta il conflitto $i(j)$.

\begin{observation}\label{obs:confC_singolo_sogg}
  Un veicolo può essere soggetto di un solo conflitto di tipo C ma oggetto di più conflitti.
\end{observation}

\begin{observation}\label{obs:grafo_albero}
  Per la costruzione del grafo di conflitti e per l'osservazione \ref{obs:confC_singolo_sogg}, il grafo di conflitti di tipo C è una foresta in cui tutti gli archi puntano verso la radice del proprio albero.
\end{observation}

\begin{proposition}\label{prop:eq_catena_foglia}
  Per costruzione dell'albero di conflitti, abbiamo che ogni cammino da una foglia a una radice corrisponde a una catena di conflitti.
\end{proposition}

\end{document}