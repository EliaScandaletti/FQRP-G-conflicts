\documentclass[../../../relazione.tex]{subfiles}

\begin{document}

Dato un problema di $n$ colonne, vogliamo calcolare la probabilità che un veicolo $i$ sia soggetto di un conflitto di tipo C.
Formalmente vogliamo calcolare
\begin{equation*}
  P(\exists j : i(j))
\end{equation*}

Per simmetria della definizione di $\mathbb{R}$ e $\mathbb{L}$, si ha
\begin{equation}\label{analisi:probC:simmRL}
  P(\exists j : i(j)\ \mid\ i \in \mathbb{R}) = P(\exists j : i(j)\ \mid\ i \in \mathbb{L})
\end{equation}

Inoltre, per la definizione di conflitto di tipo C e la definizione di $\mathbb{H}$, si ha
\begin{equation}\label{analisi:probC:probH}
  P(\exists j : i(j)\ \mid\ i \in \mathbb{H}) = 0
\end{equation}

Essendo $\mathbb{R}$, $\mathbb{L}$ e $\mathbb{H}$ partizione dell'insiemi dei veicoli, dalle equazioni (\ref{analisi:probC:simmRL}) e (\ref{analisi:probC:probH}), otteniamo
\begin{equation}\label{analisi:probC:prob2R}
  \begin{split}
    P(\exists j : i(j)) &= 2 * P(\exists j : i(j)\ \mid\ i \in \mathbb{R}) \\
    &= 2 * P(\exists j : i(j)\ \mid\ i \in \mathbb{L})
  \end{split}
\end{equation}

Scomponiamo la condizione $\exists j : i(j)$ in condizioni più piccole.
Basandoci sulla definizione (\ref{def:nodeC}), si ha
\begin{subnumcases}{
    \exists j : i(j) \iff \label{analisi:probC:condExGen} }
  j = 2\sigma(i) - i \label{analisi:probC:condExGen:a}\\
  1 \leq j \leq n \label{analisi:probC:condExGen:b}\\
  \begin{cases}\label{analisi:probC:condExGen:c}
    \sigma(j) < \sigma(i) & \text{se $i \in \mathbb{R}$} \\
    \sigma(j) > \sigma(i) & \text{se $i \in \mathbb{L}$}
  \end{cases}\\
  \begin{cases}\label{analisi:probC:condExGen:d}
    j \in \mathbb{L} & \text{se $i \in \mathbb{R}$} \\
    j \in \mathbb{R} & \text{se $i \in \mathbb{L}$}
  \end{cases}
\end{subnumcases}

Da notare che (\ref{analisi:probC:condExGen:a}) e (\ref{analisi:probC:condExGen:c}) implicano (\ref{analisi:probC:condExGen:d}).
Si assuma il caso $i \in \mathbb{R}$; il caso $i \in \mathbb{L}$ è analogo.
Per ipotesi, $i < \sigma(i)$, di conseguenza, per la (\ref{analisi:probC:condExGen:a}) $j = 2\sigma(i) - i > \sigma(i) + i - i = \sigma(i)$.
Insieme alla (\ref{analisi:probC:condExGen:c}), $j > \sigma(i) > \sigma(j) \therefore j \in \mathbb{L}$.
Da ciò si evince anche che $j \geq 1$ ($j \leq n$ nel caso $i \in \mathbb{L}$).

Si noti anche che, per le probabilità condizionate in (\ref{analisi:probC:prob2R}), possiamo considerare solo la prima (o la seconda) condizione di (\ref{analisi:probC:condExGen:c}) e (\ref{analisi:probC:condExGen:d}).

Infine, si noti che la condizione (\ref{analisi:probC:condExGen:a}) può essere portata fuori dal sistema e usata come ipotesi.

Possiamo quindi scrivere che
\begin{equation}\label{analisi:probC:condEx}
  \exists j : i(j) \iff
  \begin{cases}
    j \leq n \\
    \sigma(j) < \sigma(i)
  \end{cases}
\end{equation}
con $i \in \mathbb{R}$ e $j = 2\sigma(i) - i$.

Per l'equivalenza (\ref{analisi:probC:condEx}), si può affermare che
\begin{equation}\label{analisi:probC:chainRule}
  \begin{split}
    P(\exists j : i(j)\ \mid\ i \in \mathbb{R})
    &= P(\exists J : i(j)\ \mid\ i < \sigma(i)) \\
    &= P(j \leq n \wedge \sigma(j) < \sigma(i)\ \mid\ i < \sigma(i)) \\
    &= \frac{P(j \leq n \wedge \sigma(j) < \sigma(i) \wedge i < \sigma(i))}{P(i < \sigma(i))} \\
    &= \frac{P(\sigma(j) < \sigma(i)\ \mid\ j \leq n \wedge i < \sigma(i)) * P(j \leq n \wedge i < \sigma(i))}{P(i < \sigma(i))} \\
    &= \frac{P(\sigma(j) < \sigma(i)\ \mid\ j \leq n \wedge i < \sigma(i)) * P(j \leq n\ \mid\ i < \sigma(i)) * P(i < \sigma(i))}{P(i < \sigma(i))} \\
    &= P(\sigma(j) < \sigma(i)\ \mid\ j \leq n \wedge i < \sigma(i)) * P(j \leq n\ \mid\ i < \sigma(i))
  \end{split}
\end{equation}
con $j = 2\sigma(i) - i$.

Procediamo ora a calcolare il valore di ciascun fattore dell'equazione (\ref{analisi:probC:chainRule}).
\paragraph{$P(j \leq n\ \mid\ i < \sigma(i))$}
Sostituendo $j$, otteniamo
\[j = 2\sigma(i) - i \leq n \implies \sigma(i) \leq \frac{n+i}{2}\]
Per definizione del problema, $\sigma(i) \leq n$.
Calcoliamo la probabilità come il rapporto tra casi favorevoli e casi possibili.
I casi possibili sono $n-i$, i.e. $i < \sigma(i) \leq n$.
I casi favorevoli sono $\lfloor \frac{n-i}{2} \rfloor$, i.e. $i < \sigma(i) \leq \frac{n+i}{2}$.

Da ciò
\begin{equation}\label{analisi:probC:chainRule:part1}
  P(j \leq n\ \mid\ i < \sigma(i)) =
  \begin{cases}
    \frac{1}{2},                  & \mbox{se $i+n$ pari}    \\
    \frac{1}{2}\frac{n-i-1}{n-i}, & \mbox{se $i+n$ dispari}
  \end{cases}
\end{equation}

\paragraph{$P(\sigma(j) < \sigma(i)\ \mid\ j \leq n \wedge i < \sigma(i))$}
Dalle condizioni $j \leq n \wedge i < \sigma(i)$ sappiamo che $i < \sigma(i) \leq \frac{n+i}{2}$ per quanto visto al punto precedente.
Poiché il valore di $\sigma(j)$ è casuale e uniformemente distribuito, allora $P(\sigma(j) < \sigma(i)) = \frac{\sigma(i)-1}{n}$.

Al fine di rimuovere la dipendenza da $\sigma(i)$, bisogna risolvere una sommatoria per tutti i possibili valori di $\sigma(i)$.

\begin{equation*}
  \begin{split}
    P(\sigma(j) < \sigma(i)\ \mid\ j \leq n \wedge i < \sigma(i))
    &= \sum_{k = i+1}^{\lfloor \frac{n+1}{2} \rfloor} \frac{k-1}{n} * P(\sigma(i)=k) \\
    &= \sum_{k = i+1}^{\lfloor \frac{n+1}{2} \rfloor} \frac{k-1}{n} * \frac{1}{n} \\
    &= \frac{1}{n^2}\sum_{k = i+1}^{\lfloor \frac{n+1}{2} \rfloor}k-1 \\
    &= \frac{1}{n^2}\sum_{k = i}^{\lfloor \frac{n-1}{2} \rfloor}k \\
    &= \frac{1}{n^2}\left( \sum_{k = 1}^{\lfloor \frac{n-1}{2} \rfloor}k - \sum_{k = 1}^{i-1}k \right) \\
    &= \frac{1}{n^2}\left( \frac{\lfloor \frac{n+1}{2} \rfloor\lfloor \frac{n-1}{2} \rfloor}{2} - \frac{i(i-1)}{2} \right) \\
    &= \frac{1}{2n^2}\left( \lfloor \frac{n+1}{2} \rfloor\lfloor \frac{n-1}{2} \rfloor - i(i-1) \right) \\
  \end{split}
\end{equation*}

Distinguendo i casi $n$ pari e dispari, avremo
\begin{equation}\label{analisi:probC:chainRule:part2}
  P(\sigma(j) < \sigma(i)\ \mid\ j \leq n \wedge i < \sigma(i)) =
  \begin{cases}
    \frac{1}{2n^2}\left( \frac{n}{2} \left(\frac{n}{2}-1\right) - i(i-1) \right), & \mbox{se $n$ pari} \\
    \frac{1}{2n^2}\left( \frac{n}{2} \left(\frac{n}{2}+1\right) - i(i-1) \right), & \mbox{se $n$ dispari}
  \end{cases}
\end{equation}

Mettendo insieme i risultati ottenuti dalle equazioni (\ref{analisi:probC:chainRule}), (\ref{analisi:probC:chainRule:part1}), (\ref{analisi:probC:chainRule:part2}), otteniamo che
\begin{equation}
  P(\exists j : i(j)\ \mid\ i \in \mathbb{R}) =
  \begin{cases}
    \frac{1}{4n^2}\left( \frac{n}{2} \left(\frac{n}{2}-1\right) - i(i-1) \right), & \mbox{se $n$ pari e $i$ pari} \\
    \frac{n-i-1}{2n^2(n-i)}\left( \frac{n}{2} \left(\frac{n}{2}-1\right) - i(i-1) \right), & \mbox{se $n$ pari e $i$ dispari} \\
    \frac{1}{4n^2}\left( \frac{n}{2} \left(\frac{n}{2}+1\right) - i(i-1) \right), & \mbox{se $n$ dispari e $i$ dispari} \\
    \frac{n-i-1}{2n^2(n-i)}\left( \frac{n}{2} \left(\frac{n}{2}+1\right) - i(i-1) \right), & \mbox{se $n$ dispari e $i$ pari}
  \end{cases}
\end{equation}

\end{document}