\documentclass[../../relazione.tex]{subfiles}

\begin{document}

A termine del tirocinio, oltre alla presente relazione, vengono consegnati i seguenti file :
\begin{itemize}
    \item \textbf{cartella documenti}:
          \begin{itemize}
              \item \textbf{relazione.tex}, sorgente principale della relazione;
              \item \textbf{preamble.tex}, preambolo \LaTeX della relazione;
              \item \textbf{cartella cache}, vuota ma necessaria per salvare i grafici ed evitare tempi di compilazione troppo lunghi ad ogni ricompilazione;
              \item \textbf{cartella data}, contiene i dati ottenuti dall'esecuzione dei programmi;
              \item \textbf{cartella sections}, contiene i vari file \LaTeX della relazione;
          \end{itemize}
    \item \textbf{cartella pseudocode}, contiene pseudocodice che è stato utilizzato come base per il codice vero e proprio;
    \item \textbf{cartella src}, contiene il codice sorgente dei programmi:
          \begin{itemize}
              \item \textbf{cartella include}, contiene i file header utilizzati;
              \item \textbf{Makefile}, contiene le istruzioni per la compilazione degli eseguibili;
              \item altre cartelle e file contenenti i sorgenti dei programmi;
          \end{itemize}
    \item \textbf{convert.sh}, file bash che converte i file di dati generati dai programmi in file di dati utili alla creazione dei grafici;
    \item \textbf{run.sh}, file bash di utilità per far eseguire tutti i programmi nella configurazione utilizzata.
\end{itemize}

Inoltre, si allegano i file contenenti il benchmark di istanze generato.

\end{document}