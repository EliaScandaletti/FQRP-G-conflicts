\documentclass[../main.tex]{subfiles}
\graphicspath{{\subfix{../images/}}}

\begin{document}

In caso il conflitto avvenga su un arco, si parla di conflitto di arco. In caso il conflitto avvenga su un nodo, si parla di conflitto di nodo.

I conflitti di nodo sono ulteriormente divisi in conflitto di tipo A, conflitti di tipo B e conflitti di tipo C.

\paragraph{Conflitti di arco}
Dati due veicoli $i$ e $j$, questi sono in conflitto di arco se e solo se le seguenti condizioni sono rispettate:
\begin{subnumcases}{\label{def:arc}}
  i \in \mathbb{R}\label{def:arc:a} \\
  j \in \mathbb{L}\label{def:arc:b} \\
  j - i = 2n + 1,\ n \in \mathbb{N}\label{def:arc:c} \\
  \sigma(j) < \frac{i+j}{2} < \sigma(i)\label{def:arc:d}
\end{subnumcases}
In questo caso si potrebbe generare una collisione negli archi (\ref{def:arc:c}) tra la colonna $\frac{i+j-1}{2}$ e la colonna $\frac{i+j+1}{2}$ poiché i due veicoli ci passano in contemporanea (\ref{def:arc:d}) e in direzioni opposte (\ref{def:arc:a}, \ref{def:arc:b}).

\paragraph{Conflitti di nodo}
Dati due veicoli $i$ e $j$, questi possono essere in un conflitto di tipo A, B o C.

\subparagraph{Conflitti di nodo di tipo A}
I conflitti di tipo A sono un particolare caso di conflitti di nodo. In particolare, un conflitto è di tipo A se e solo se rispetta le seguenti condizioni:
\begin{subnumcases}{\label{def:nodeA}}
  \min\{j, \sigma(j)\} < \sigma(i) < \max\{j, \sigma(j)\}\label{def:nodeA:a} \\
  \left|i-\sigma(i)\right| < \left|j-\sigma(i)\right|\label{def:nodeA:b}
\end{subnumcases}

\subparagraph{Conflitti di nodo di tipo B}
I conflitti di tipo B sono un particolare caso di conflitti di nodo. In particolare, un conflitto è di tipo B se e solo se rispetta le seguenti condizioni:
\begin{subnumcases}{\label{def:nodeB}}
  i \in \mathbb{R}\label{def:nodeB:a} \\
  j \in \mathbb{L}\label{def:nodeB:b} \\
  j - i = 2n,\ n \in \mathbb{N}\label{def:nodeB:c} \\
  \sigma(j) < \frac{i+j}{2} < \sigma(i)\label{def:nodeB:d}
\end{subnumcases}

\end{document}