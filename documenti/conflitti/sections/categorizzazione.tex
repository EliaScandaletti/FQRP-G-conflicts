\documentclass[../main.tex]{subfiles}
\graphicspath{{\subfix{../images/}}}

\begin{document}

In caso il conflitto avvenga su un arco, si parla di conflitto di arco. In caso il conflitto avvenga su un nodo, si parla di conflitto di nodo.

I conflitti di nodo sono ulteriormente divisi in conflitto di tipo A, conflitti di tipo B e conflitti di tipo C.

\paragraph{Conflitti di arco}
Dati due veicoli $i$ e $j$, questi sono in conflitto di arco se e solo se le seguenti condizioni sono rispettate:
\begin{subnumcases}{\label{cond:arc}}
  i \in \mathbb{R}\label{cond:arc:a} \\
  j \in \mathbb{L}\label{cond:arc:b} \\
  j - i = 2n + 1,\ n \in \mathbb{N}\label{cond:arc:c} \\
  \sigma(j) < \frac{i+j}{2} < \sigma(i)\label{cond:arc:d}
\end{subnumcases}
In questo caso si potrebbe generare una collisione negli archi (\ref{cond:arc:c}) tra la colonna $\frac{i+j-1}{2}$ e la colonna $\frac{i+j+1}{2}$ poiché i due veicoli ci passano in contemporanea (\ref{cond:arc:d}) e in direzioni opposte (\ref{cond:arc:a}, \ref{cond:arc:b}).

\paragraph{Conflitti di nodo}
Dati due veicoli $i$ e $j$, questi possono essere in un conflitto di tipo A, B o C.

\subparagraph{Conflitti di nodo di tipo A}
I conflitti di tipo A sono un particolare caso di conflitti di nodo. In particolare, un conflitto è di tipo A se e solo se rispetta le seguenti condizioni:
\begin{subnumcases}{\label{cond:nodeA}}
  \min\{j, \sigma(j)\} < \sigma(i) < \max\{j, \sigma(j)\}\label{cond:nodeA:a} \\
  \left|i-\sigma(i)\right| < \left|j-\sigma(i)\right|\label{cond:nodeA:b}
\end{subnumcases}
In questo caso potrebbe generarsi una collisione nel nodo $(m, \sigma(i))$, dove $m$ è l'ultimo livello della griglia, poiché:
\begin{itemize}
  \item è nel percorso di $i$, poiché è la sua destinazione;
  \item è in una colonna attraversata da $j$ (\ref{cond:nodeA:a});
  \item $i$ ci arriva prima di $j$ e ci si ferma (\ref{cond:nodeA:b}).
\end{itemize}

\subparagraph{Conflitti di nodo di tipo B}
I conflitti di tipo B sono un particolare caso di conflitti di nodo. In particolare, un conflitto è di tipo B se e solo se rispetta le seguenti condizioni:
\begin{subnumcases}{\label{cond:nodeB}}
  i \in \mathbb{R}\label{cond:nodeB:a} \\
  j \in \mathbb{L}\label{cond:nodeB:b} \\
  j - i = 2n,\ n \in \mathbb{N}\label{cond:nodeB:c} \\
  \sigma(j) < \frac{i+j}{2} < \sigma(i)\label{cond:nodeB:d}
\end{subnumcases}
In questo caso potrebbe generarsi una collisione nei nodi della colonna $\frac{i+j}{2}$ poiché tale colonna viene raggiunta dai due veicoli in contemporanea.

\end{document}