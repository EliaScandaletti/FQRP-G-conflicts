\documentclass[../../main.tex]{subfiles}
\graphicspath{{\subfix{../../images/}}}

\begin{document}

Dato un problema di $n$ colonne, vogliamo calcolare la probabilità che un veicolo $i$ appartenga a $\mathbb{R}$, $\mathbb{L}$ o $\mathbb{H}$.

Iniziamo osservando che gli insiemi $\mathbb{R}$, $\mathbb{L}$ ed $\mathbb{H}$ sono una partizione dell'insieme dei veicoli.
Da questo segue che
\begin{equation}\label{analisi:part}
  P(i \in \mathbb{R}) + P(i \in \mathbb{L}) + P(i \in \mathbb{H}) = 1
\end{equation}

Inoltre, per la simmetria della definizione di $\mathbb{R}$ e $\mathbb{L}$, si ha
\begin{equation}\label{analisi:simmRL}
  P(i \in \mathbb{R}) = P(i \in \mathbb{L})
\end{equation}

Calcolare $\mathbb{H}$ è banale.
Si ha
\begin{equation}\label{analisi:probH}
  P(i \in \mathbb{H}) = \frac{1}{n}
\end{equation}

Dalle equazioni (\ref{analisi:part}), (\ref{analisi:simmRL}) e (\ref{analisi:probH}), si ottiene
\begin{equation}
  P(i \in \mathbb{R}) = P(i \in \mathbb{L}) = \frac{n-1}{2n}
\end{equation}

\end{document}