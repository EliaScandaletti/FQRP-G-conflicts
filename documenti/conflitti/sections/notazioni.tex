\documentclass[../main.tex]{subfiles}
\graphicspath{{\subfix{../images/}}}

\begin{document}

Si dice che i veicoli $i$ e $j$ sono in conflitto qualora esistano i percorsi $\pi_i$ e $\pi_j$, rispettivamente dei veicoli $i$ e $j$, tali per cui attraversano lo stesso arco o lo stesso nodo nello stesso momento.

Indichiamo con $\sigma(i)$ la colonna di destinazione del veicoli $i$. La colonna di partenza del veicolo $i$ è sempre $i$.

In base alle posizioni relative delle colonne di partenza e arrivo, possiamo dividere i veicoli in tre insiemi complementari:
\begin{itemize}
  \item $\mathbb{H} \coloneqq \{i | \sigma(i) = i\}$;
  \item $\mathbb{R} \coloneqq \{i | \sigma(i) > i\}$;
  \item $\mathbb{L} \coloneqq \{i | \sigma(i) < i\}$.
\end{itemize}



\end{document}