\documentclass[../main.tex]{subfiles}

\begin{document}

\subsection{Introduzione}
In questa sezione vengono esposti i risultati numerici ottenuti.

Per ogni metrica misurata, per ogni generatore utilizzato, viene proposta la distribuzione in funzione della dimensione del problema.

\subsection{Conflitti di arco}
\makefigure{Conflitti di arco}{arcType}

\subsection{Conflitti di tipo A}
\makefigure{Conflitti di tipo A}{AType}

\subsection{Conflitti di tipo B}
\makefigure{Conflitti di tipo B}{BType}

\subsection{Conflitti di tipo C}
\subsubsection{Numero di conflitti}
\makefigure{Numero di conflitti}{CGraph_arcs_num}

\subsubsection{Numero di catene}
\makefigure{Numero di catene}{CGraph_chain_num}

\subsubsection{Lunghezza della catena massima}
\makefigure{Lunghezza della catena massima}{CGraph_max_length}

\subsubsection{Numero di alberi di conflitto}
\makefigure{Numero di alberi di conflitto}{CGraph_tree_num}

\subsubsection{Numero di veicoli coinvolti}
\makefigure{Numero di veicoli coinvolti}{CGraph_vehicles_num}

\subsection{Conflitti di tipo misto}
\subsubsection{Numero di conflitti}
\makefigure{Numero di conflitti}{MForest_edges_num}

\subsubsection{Dimensione dell'albero di conflitti più grande}
\makefigure{Dimensione dell'albero di conflitti più grande}{MForest_max_tree_size}

\subsubsection{Numero di veicoli coinvolti}
\makefigure{Numero di veicoli coinvolti}{MForest_nodes_num}

\subsubsection{Numero di alberi}
\makefigure{Numero di alberi}{MForest_tree_num}

\end{document}